\documentclass[a4paper,12pt]{article}
\usepackage{xcolor}
\usepackage{listings}
\lstset{basicstyle=\ttfamily,
  showstringspaces=false,
  commentstyle=\color{red},
  keywordstyle=\color{blue}
}

\title{CANTHERM}
\author{Sandeep Sharma and James Smith}
\begin{document}
 \maketitle
\tableofcontents
\section{Introduction}
CANTHERM is written in python and calculates the thermodynamic properties of stable molecules and rate coefficients for reactions. The types of calculations it can perform are
\begin{itemize}
 \item Calculates enthalpy, entropy and heat capacities using Rigid Rotor Harmonic Oscillator approximation with correction for hindered rotors.
\item Calculates the high pressure rate coefficients for a unimolecular reaction.
\end{itemize}

The manual first attempts to give some background theory and then goes on to give the instructions for making the input file. Finally a couple of example input files along with the output are given.

\subsection{Dependencies}
\begin{enumerate}
  \item Numpy
  \item Scipy
  \item Matplotlib
\end{enumerate}

\section{Theory}
Rigid Rotor Harmonic oscillator assumes separability of different modes of vibrations and interal/external rotations in a molecule. Each mode is then treated separately to calculate the partition function and subsequently the thermodynamic quantities from the partition functions. We treat each of the modes separately.
\subsection{Translation}
The partition function for translation is given in Equation~\ref{eq:tr1}. The resulting thermodynamic functions are given in Equations~\ref{eq:tr2}
\begin{equation}
 Q_{tr} = \left(\frac{2\pi m k_bT}{h^2}\right)^{3/2}V
\label{eq:tr1}
\end{equation}

\begin{eqnarray}
 S%&=&\frac{\partial}{\partial T}\left(k_bT\ln Q_{tr}\right)\nonumber\\
&=&k_b \ln\left[\left(\frac{2\pi m k_bT}{h^2}\right)^{3/2} \frac{k_bTe^{5/2}}{p}\right]\nonumber\\
c_p&=& \frac{5}{2}k_b \nonumber \\
\Delta H&=& \frac{5}{2}k_bT
\label{eq:tr2}
\end{eqnarray}

\subsection{Vibrations}
Vibrational partition functions and the corresponding thermodynamic functions can be calculated exactly assuming that the vibrations behave like a harmonic oscillator. If the frequency of the $i^{th}$ vibration is given by $/nu_i$ then the quantities of interest are given in Equation~\ref{eq:vi1}. Note that for the expression of $\Delta H$ we have not included the zero point energy because we assume that the ground state is the electronic energy plus the zero point energy. It is the same reason that the $\exp{(-h\nu_i/2k_bT)}$ term is missing from the numerator of the partition function. The reference energy has no effect on the heat capacity and entropy and their formulas will remain the same.
\begin{eqnarray}
Q_{ho} &=& \frac{1}{1-\exp{(-h\nu_i/k_bT)}} \nonumber \\
S &=& -k_b \ln\left(1-\exp{(-h\nu_i/k_bT)}\right)+\frac{h\nu_i}{T}\frac{1}{\exp{(h\nu_i/k_bT)}-1}\nonumber \\
c_p &=& k_b\left(\frac{h\nu_i}{k_bT}\right)^2 \frac{\exp{(h\nu_i/k_bT)}}{\left(\exp{(h\nu_i/k_bT)}-1\right)^2}\nonumber \\
\Delta H &=& \frac{h\nu_i}{\exp{(h\nu_i/k_bT)}-1}
\label{eq:vi1}
\end{eqnarray}

\subsection{Rotations}
For the molecule without internal rotors only the external rotors contribute to the rotational partition function. The partition functions and thermodynamic quantities for these rotors is relatively straight forward to calculate and are given in Equations~\ref{eq:r1}.
\begin{eqnarray}
 Q_{e,r}&=&\frac{\pi^{1/2}}{\sigma_{ext}}\prod_i \left(\frac{8\pi^2I_ik_bT}{ h^2}\right)^{1/2} \nonumber \\
S&=& k_b \left[\frac{\pi^{1/2}}{\sigma_{ext}} \ln \left(\prod_i \frac{8\pi^2I_ik_bT}{h^2}\right)^{1/2} +\frac{3}{2}\right] \nonumber \\
c_p&=& \frac{3}{2}k_b\nonumber \\
\Delta H &=& \frac{3}{2}k_bT
\label{eq:r1}
\end{eqnarray}

When a molecule has internal rotors then the external and internal rotors are coupled to each other and the combined partition function is calculted using the semi-classical Pitzer-Gwinn approximation given in Equation~\ref{eq:1}, where $Q_{r}$ is the total rotational partition function including the internal and external modes, $Q_{r,cl}$ is the classical partition function for the combined internal and external rotations calculated using the phase space integral, $Q_{ho,q}$ is the quantum partition function of the vibrational modes corresponding to the hindered rotor and $Q_{ho,cl}$ is the classical partition function of the vibrational modes corresponding to the hindered rotor.
\begin{equation}
 Q_{r} = Q_{r,cl}\frac{Q_{ho,q}}{Q_{ho,cl}}
\label{eq:1}
\end{equation}

The expressions for each of the partition functions is given in Equation~\ref{eq:2} and Equation~\ref{eq:vi1}.
\begin{eqnarray}
 Q_{r,cl} &=& \frac{\pi^{1/2}}{\sigma_{ext}}\left(\frac{8\pi^2k_bT}{ h^2}\right)^{3/2} \left( \frac{2 \pi k_b T}{h^2}\right)^{n/2} \frac{1}{\prod_i \sigma_i}\int_{\Phi} [D]^{1/2} \exp{(-V(\Phi)/k_bT)} d\Phi \nonumber \\
Q_{ho,cl} &=& \prod_i \frac{k_bT}{h\nu_i}
\label{eq:2}
\end{eqnarray}

For free rotors we just substitute $V(\Phi)=0$ in the expression for $Q_{r,cl}$ to get the expression shown in Equation~\ref{eq:3}.
\begin{equation}
 Q_{fr,cl} = \frac{\pi^{1/2}}{\sigma_{ext}}\left(\frac{8\pi^2k_bT}{ h^2}\right)^{3/2} \left( \frac{2 \pi k_b T}{h^2}\right)^{n/2} \frac{1}{\prod_i \sigma_i}\int_{\Phi} [D]^{1/2} d\Phi
\label{eq:3}
\end{equation}

Usually when only free rotors are present in the molecule one would just pick the geometry of the most stable confirmer and calculate the kinetic energy matrix $D$. In Cantherm we use the procedure described in 1949 paper of Pitzer and Gwinn which is described as $I^{m=5}$ in the paper by East and Radom. This method is completely accurate for a molecule which has rigid frames moving with respect to each other. The rigid frame approximation ignores the rotational-vibrational coupling in the molecule.

To calculate the phase space integral in Equation~\ref{eq:2} there are a few approximations that can be made. If we assume that the potential $V(\Phi)$ can be broken up as a sum of potentials in different phase angles i.e. $V(\Phi) = V_1(\phi_1)+V_2(\phi_2)...$ then the multidimensional integral can be broken into a product of single dimensional integrals. These integrals can then separately be evaluated using trapezoidal rule or monte-carlo integration. For each phase a complete scan is performed and then the scan potential is fit to a fourier series of the form $V_i(\phi_i)=\sum_m \cos(m\phi_i)+\sin(m\phi_i)$. The fitted parameters can then be used to calculate the integrals.

For some rare cases the separation of variables does not work very well. In this case the general approach is to calculate the multidimensional integral numerically using monte-carlo integration technique. Each of the phase angle $\phi_i$ is chosen randomly and for that chosen phase angle the energy of the molecule is evaluated using some ab-initio method.

% \subsection{Identification of internal rotation modes}
% For small molecules the identification of internal modes of rotation is fairly straight forward. If a viewing package is used one of the normal modes usually very strongly resembles an internal rotation. But for larger molecules the situation becomes complicated the normal modes of vibrations have many internal rotational modes along with other modes including bending and stretching modes mixed in. In this case identifying the correct normal mode to remove and treat it as hindered rotor can be tricky. For such cases evaluating the force constant matrix and projecting out the force constants along the internal rotation coordinate can give us a more accurate picture.
%
% Let us assume that atom 1 and atom 2 are the two pivot atoms for an internal rotor $i$. Also assume that a list of atoms $k$ is attached to pivot atom 1 and a list of atoms $l$ is attached to pivot atom 2. Then the instantaneous displacement vector for an atom $kj$ in list $k$ is given by the vector in Equation~\ref{eq:dis}, where $p_n$ is the position vector of the $n^{th}$ atom.
% \begin{equation}
% s_{kj} = \frac{(p_{kj}-p_1) \times (p_2-p_1)}{\left|p_2-p_1\right|}
% \label{eq:dis}
% \end{equation}
%
% Similar displacement vectors can be calculated for each atom and then the $i^{th}$ internal rotation in cartesian coordinate is given by vector $v_i = [s_1 s_2 s_3 ... s_n]^T$ where $s_i$ is the instantaneous displacement vector give in Equation~\ref{eq:dis} and $n$ is the total number of atoms in the molecule. One the vectors $v_i$ corresponding to all the internal rotors are calculated an orthonormal set $w_i$ is generated from these vectors. The projection vector $P$ can be generated from these orthonormal set $w_i$ as shown in Equation~\ref{eq:pr} where matrix $W=[w_1 w_2..]$ has column vectors $w_i$.
%
% \begin{equation}
%  P = WW^T
% \label{eq:pr}
% \end{equation}
%
% If the complete force constant matrix is $Fc$ then the new force constant matrix $Fcr$ from which the force constants along the internal rotation vectors are removed are given by Equation~\ref{eq:frc} and has $m$ less non-zero (or nearly non-zero) eigenvalues as $Fc$, where $m$ is the number of internal rotors.
%
% \begin{equation}
%  Fcr = (I-P)Fc(I-P)
% \label{eq:frc}
% \end{equation}
%
% This force matrix $Fcr$ can be converted into mass-weighted cartesian coordinates and diagonalized to get the eigenvalues from which the vibrational frequencies can be calculated.

\section{Input File}
\subsection{Type of Calculation}
The first line of the file contains the keywords describing the job type to be performed, the two options are \textbf{Thermo} or \textbf{Reac}. As the name suggests the thermo keyword is used to calculated the thermodynamic quantities including entropy, heat capacity and thermal correction for a given molecule. The Rate keyword is used to calculate the rate coefficients for a reaction and also the fitted Arrhenius parameters, note that the thermodynamic info is also written to the output in this instance.

If the keyword \textbf{Rate} is used then the next line has to describe whether the reaction is unimolecular or bimolecular by giving keywords \textbf{Unimol} or \textbf{Bimol} respectively. Right now only \textbf{Unimol} is supported.

\subsection{Temperatures for Thermodynamics}
The next line gives the range of Temperatures at which the thermochemistry or the rates are to be evaluated.There are two possible options here:

First, you can specify \textbf{TLIST} and then on the next line specify the number of temperature values you would like to work with \textit{and then on the next line} write each temperature separated by a space. This the preferred method.

Second, the user can user the keyword \textbf{TRANGE} to specify an evenly space set of temperatures.
The syntax is \textbf{TRANGE} and then on the next line \textbf{$T_0$ $dT$ $n$}. Here $T_i = T_0 + i*dT$ where $T_i$ is the i$^{th}$ temperature, $dT$ is the positive change between the temperature values, and $n$ is the number of temperates you want to work with.

\subsection{Scale and Root Dir}
On the next line we specify the scale factor used for the frequencies after the keyword \textbf{SCALE}.
Then on the next line we specify the root directory for all of the files that will be used by the program. If all the files you wish to use are in you current working directly you should write \textbf{ROOTDIR} followed by a period. If they are somewhere else on your computer please specify where with an absolute path and \textit{do not leave a slash on the end of your path!}

\subsection{Molecules}
The first line of your molecule section must by \textbf{MOL} followed by a label for the molecule. This will be used to identify the plot of the rotational PES which are saved in the current working directory (if the hindered rotor treatment is applied). The next line must be either \textbf{LINEAR} or \textbf{NONLINEAR}. \underline{All of the attributes set or specified in this section will have to be specified for each molecule.} It is important to note that all of these keywords must appear \underline{IN ORDER}. Make sure that all of the files you reference from your input have the atoms ordered consistently!

\subsubsection{Geometry}
The user can specify the geometry of a molecule with the keyword \textbf{GEOM} in two ways:

First, the user can write \textbf{FILE} followed by the path (not counting the portion specified with the \textbf{ROOTDIR} keyword). This will read the optimized geometry from the output file of the electronic structure program.
At the time of writing, the software supports only Gaussian 09 and 16 output files for geometry. The full line might look something like \textit{GEOM FILE gaussian-output}. \underline{\textit{This is the preferred method.}}

Second, the user can specify the geometry explicitly in the input file. On a new line, after the \textbf{GEOM} keyword, the user must specify the number of atoms in the molecule and then on the subsequent lines write the number of the atom in the list of all atoms in the molecule, the atomic number, the symbol, and then the x, y, and z coordinates \underline{in angstroms}.

\subsubsection{Harmonic Frequencies}
The user can assign the harmonic frequencies of the molecule with the keyword \textbf{FREQ} in two ways, similar to the manner in which we specify the geometry:

First, the user can write \textbf{FILE} followed by the path (not counting the portion specified with the \textbf{ROOTDIR} keyword). This will read the harmonic frequencies from the output file of the electronic structure program.
At the time of writing, the software supports only Gaussian 09 and 16 output files for frequencies. The full line might look something like \textit{FREQ FILE gaussian-output}. \underline{\textit{This is the preferred method.}}

Second, the user can specify the frequencies explicitly in the input file. On a new line, after the \textbf{FREQ} keyword, the user must specify the number of frequencies for the molecules and on the subsequent lines write the frequencies in \underline{inverse centimeters}.

\subsubsection{Energy}
The user can assign the energy of the molecule with the keyword \textbf{ENERGY} in two ways, similar to the manner in which we specify the geometry and frequencies:

First, the user can write \textbf{FILE} followed by the path (not counting the portion specified with the \textbf{ROOTDIR} keyword). This will read the energy from the output file of the electronic structure program.
At the time of writing, the software supports only Gaussian 09 and 16 output files for frequencies. The full line might look something like \textit{ENERGY FILE gaussian-output}. \underline{\textit{This is the preferred method.}}

Second, the user can specify the energy explicitly in the input file. After the \textbf{ENERGY} keyword, the user can write the energy in \underline{hartrees}.

\subsubsection{Misc. Molecular Keywords}
The user must also specify the external symmetry of the molecule after the keyword \textbf{EXTSYM}. On the next line the user must write \textbf{NELEC 2}.

\subsubsection{Hindered Rotors}
If no hindered rotors approximations are desired simply write \textbf{ROTORS} followed by 0.

If the user does wish to incorporate hindered rotor calculations into the thermodynamics (and possibly kinetics) calculations then they should specify the number of hindered rotors after the \textbf{ROTORS} keyword.

On the next line the user must then list the rotor PES scan files (keeping in mind the path stored using the \textbf{ROOTDIR}). At the time of writing, the software supports only Gaussian 09 and 16 output files for rotational scans.

On the following line, the user must specify the symmetry of each rotor after the keyword \textbf{ROTORSYM} and must do so in the order that the files appear above.

Finally, the user must specify the frequency most closely associated with the hindered rotation, after the keyword \textbf{ROTOFREQ}. These frequency will be removed from the list of harmonic frequencies used to calculate the vibrational partition function.

\subsection{Kinetics}
If you would like to calculate kinetics for a unimolecular reaction then you may include another molecule in your input file. The software treats the first molecule as the ground state and the second as the transition state. The procedure to specify the attributes of the transition state is the same as described above.

\section{Running Cantherm}
The user can run Cantherm either programmatically or from the command line. At this time we recommend running from the command line with a text file input. The user can execute a job with the following command in a bash shell:

\begin{lstlisting}[language=bash]
python /path/to/CANTHERM/cantherm/main.py input output
\end{lstlisting}

If running Cantherm in the command line you must specify the desired output file.

\end{document}
